
\begin{document}
%
% paper title
% can use linebreaks \\ within to get better formatting as desired
\title{Model Driven Development in Power Electronics}


% author names and affiliations
% use a multiple column layout for up to three different
% affiliations
%\author{
%\IEEEauthorblockN{David Kiss}
%\IEEEauthorblockA{School of Electrical and\\Computer Engineering\\
%Georgia Institute of Technology\\
%Atlanta, Georgia 30332--0250\\
%Email: http://www.michaelshell.org/contact.html}
%\and
%\IEEEauthorblockN{Homer Simpson}
%\IEEEauthorblockA{Twentieth Century Fox\\
%Springfield, USA\\
%Email: homer@thesimpsons.com}
%\and
%\IEEEauthorblockN{James Kirk\\ and Montgomery Scott}
%\IEEEauthorblockA{Starfleet Academy\\
%San Francisco, California 96678-2391\\
%Telephone: (800) 555--1212\\
%Fax: (888) 555--1212}}

% conference papers do not typically use \thanks and this command
% is locked out in conference mode. If really needed, such as for
% the acknowledgment of grants, issue a \IEEEoverridecommandlockouts
% after \documentclass

% for over three affiliations, or if they all won't fit within the width
% of the page, use this alternative format:
% 
\author{\IEEEauthorblockN{David Kiss\IEEEauthorrefmark{1},
Adam Halmos\IEEEauthorrefmark{1},
Gergely Kalman\IEEEauthorrefmark{1}, 
Istvan Varjasi dr.\IEEEauthorrefmark{2} and
Zoltan Suto dr.\IEEEauthorrefmark{2}}
\IEEEauthorblockA{\IEEEauthorrefmark{2}Budapest University of Technology and Economics\\
Deparment of Automatization and Applied Informatics}
\IEEEauthorblockA{\IEEEauthorrefmark{1}Hyundai Technologies Center Hungary Ltd., Budapest, Hungary\\
Email: dkiss@h-tec.hu; ahalmos@h-tec.hu}
}





% use for special paper notices
%\IEEEspecialpapernotice{(Invited Paper)}




% make the title area
\maketitle


\begin{abstract}
%\boldmath
Nowadays power converters are gaining more and more importance due to the increasing demand of energy efficient solutions in industrial applications, renewable energy sources and electrical transportation. To meet this new demand, development cycles must be shortened, without compromising the quality of the control software. This goal can be achieved with model driven development, and it also can make the process more agile. In the presented method, we are using MATLAB Simulink to implement the control related software components. These software modules, such as control loops and state machines can be functionally tested in the Simulink environment, with much better observability than the real embedded software. The embedded code is generated from these models. The embedded code only contains the framework, which is providing a interface between the model and the control peripherals and the real time operating system. In this paper we are discussing the process in detail on a frequency converter firmware as example.
\end{abstract}
% IEEEtran.cls defaults to using nonbold math in the Abstract.
% This preserves the distinction between vectors and scalars. However,
% if the journal you are submitting to favors bold math in the abstract,
% then you can use LaTeX's standard command \boldmath at the very start
% of the abstract to achieve this. Many IEEE journals frown on math
% in the abstract anyway.

% Note that keywords are not normally used for peerreview papers.
\begin{IEEEkeywords}
IYCE, Power Electronics, MATLAB, Simulink, Motor Control, VFD, Model Driven Development
\end{IEEEkeywords}






% For peer review papers, you can put extra information on the cover
% page as needed:
% \ifCLASSOPTIONpeerreview
% \begin{center} \bfseries EDICS Category: 3-BBND \end{center}
% \fi
%
% For peerreview papers, this IEEEtran command inserts a page break and
% creates the second title. It will be ignored for other modes.
\IEEEpeerreviewmaketitle



\section{Introduction}
A modern VFD system must comply to many customer requiroment. There are also many industrial standard such as power factor regulations and EMC limitations wich are needed to control by software. The computation power of the embeddedd microcontrollers are capeble to achive this kind of sophistication, but the this power needed to be utilized by the programmer. The development cycles also getting shorter and shortet to met the constantly changing market demand. Therfore a modular, reusable and scalable control software need to be implemented. Modern software tools such as MATLAB and Simulink are very well suited to this task. The Embedded Coder toolbox generates the C code, wich can be complied in the embedded processor.

In the II. chapter we are going to discuss the embedded framework, wich provides the connection between the Simulink model and the real microcontroller outputs and inputs. The III. chapter will present a short insight in the Simulink model, and present the challanges you can face during the development. The IV. chapter will provide a case study about the whole method. The example will be a IGBT thermal model. 


\section{Embedded Framework}
Well, the way they make shows is, they make one show. That show's called a pilot. Then they show that show to the people who make shows, and on the strength of that one show they decide if they're going to make more shows. Some pilots get picked and become television programs. Some don't, become nothing. She starred in one of the ones that became nothing.

The path of the righteous man is beset on all sides by the iniquities of the selfish and the tyranny of evil men. Blessed is he who, in the name of charity and good will, shepherds the weak through the valley of darkness, for he is truly his brother's keeper and the finder of lost children. And I will strike down upon thee with great vengeance and furious anger those who would attempt to poison and destroy My brothers. And you will know My name is the Lord when I lay My vengeance upon thee.

Normally, both your asses would be dead as fucking fried chicken, but you happen to pull this shit while I'm in a transitional period so I don't wanna kill you, I wanna help you. But I can't give you this case, it don't belong to me. Besides, I've already been through too much shit this morning over this case to hand it over to your dumb ass.
\section{Simulink Model}
Do you see any Teletubbies in here? Do you see a slender plastic tag clipped to my shirt with my name printed on it? Do you see a little Asian child with a blank expression on his face sitting outside on a mechanical helicopter that shakes when you put quarters in it? No? Well, that's what you see at a toy store. And you must think you're in a toy store, because you're here shopping for an infant named Jeb.
\section{Case Study: IGBT Thermal model}
You think water moves fast? You should see ice. It moves like it has a mind. Like it knows it killed the world once and got a taste for murder. After the avalanche, it took us a week to climb out. Now, I don't know exactly when we turned on each other, but I know that seven of us survived the slide... and only five made it out. Now we took an oath, that I'm breaking now. We said we'd say it was the snow that killed the other two, but it wasn't. Nature is lethal but it doesn't hold a candle to man.

% needed in second column of first page if using \IEEEpubid
%\IEEEpubidadjcol

% An example of a floating figure using the graphicx package.
% Note that \label must occur AFTER (or within) \caption.
% For figures, \caption should occur after the \includegraphics.
% Note that IEEEtran v1.7 and later has special internal code that
% is designed to preserve the operation of \label within \caption
% even when the captionsoff option is in effect. However, because
% of issues like this, it may be the safest practice to put all your
% \label just after \caption rather than within \caption{}.
%
% Reminder: the "draftcls" or "draftclsnofoot", not "draft", class
% option should be used if it is desired that the figures are to be
% displayed while in draft mode.
%
%\begin{figure}[!t]
%\centering
%\includegraphics[width=2.5in]{myfigure}
% where an .eps filename suffix will be assumed under latex, 
% and a .pdf suffix will be assumed for pdflatex; or what has been declared
% via \DeclareGraphicsExtensions.
%\caption{Simulation Results}
%\label{fig_sim}
%\end{figure}

% Note that IEEE typically puts floats only at the top, even when this
% results in a large percentage of a column being occupied by floats.


% An example of a double column floating figure using two subfigures.
% (The subfig.sty package must be loaded for this to work.)
% The subfigure \label commands are set within each subfloat command, the
% \label for the overall figure must come after \caption.
% \hfil must be used as a separator to get equal spacing.
% The subfigure.sty package works much the same way, except \subfigure is
% used instead of \subfloat.
%
%\begin{figure*}[!t]
%\centerline{\subfloat[Case I]\includegraphics[width=2.5in]{subfigcase1}%
%\label{fig_first_case}}
%\hfil
%\subfloat[Case II]{\includegraphics[width=2.5in]{subfigcase2}%
%\label{fig_second_case}}}
%\caption{Simulation results}
%\label{fig_sim}
%\end{figure*}
%
% Note that often IEEE papers with subfigures do not employ subfigure
% captions (using the optional argument to \subfloat), but instead will
% reference/describe all of them (a), (b), etc., within the main caption.


% An example of a floating table. Note that, for IEEE style tables, the 
% \caption command should come BEFORE the table. Table text will default to
% \footnotesize as IEEE normally uses this smaller font for tables.
% The \label must come after \caption as always.
%
%\begin{table}[!t]
%% increase table row spacing, adjust to taste
%\renewcommand{\arraystretch}{1.3}
% if using array.sty, it might be a good idea to tweak the value of
% \extrarowheight as needed to properly center the text within the cells
%\caption{An Example of a Table}
%\label{table_example}
%\centering
%% Some packages, such as MDW tools, offer better commands for making tables
%% than the plain LaTeX2e tabular which is used here.
%\begin{tabular}{|c||c|}
%\hline
%One & Two\\
%\hline
%Three & Four\\
%\hline
%\end{tabular}
%\end{table}


% Note that IEEE does not put floats in the very first column - or typically
% anywhere on the first page for that matter. Also, in-text middle ("here")
% positioning is not used. Most IEEE journals use top floats exclusively.
% Note that, LaTeX2e, unlike IEEE journals, places footnotes above bottom
% floats. This can be corrected via the \fnbelowfloat command of the
% stfloats package.



\section{Conclusion}
\blindtext





% if have a single appendix:
%\appendix[Proof of the Zonklar Equations]
% or
%\appendix  % for no appendix heading
% do not use \section anymore after \appendix, only \section*
% is possibly needed

% use appendices with more than one appendix
% then use \section to start each appendix
% you must declare a \section before using any
% \subsection or using \label (\appendices by itself
% starts a section numbered zero.)
%


\appendices
\section{Proof of the First Zonklar Equation}
\blindtext

% use section* for acknowledgement
\section*{Acknowledgment}


The authors would like to thank...


% Can use something like this to put references on a page
% by themselves when using endfloat and the captionsoff option.
\ifCLASSOPTIONcaptionsoff
  \newpage
\fi



% trigger a \newpage just before the given reference
% number - used to balance the columns on the last page
% adjust value as needed - may need to be readjusted if
% the document is modified later
%\IEEEtriggeratref{8}
% The "triggered" command can be changed if desired:
%\IEEEtriggercmd{\enlargethispage{-5in}}

% references section

% can use a bibliography generated by BibTeX as a .bbl file
% BibTeX documentation can be easily obtained at:
% http://www.ctan.org/tex-archive/biblio/bibtex/contrib/doc/
% The IEEEtran BibTeX style support page is at:
% http://www.michaelshell.org/tex/ieeetran/bibtex/
%\bibliographystyle{IEEEtran}
% argument is your BibTeX string definitions and bibliography database(s)
%\bibliography{IEEEabrv,../bib/paper}
%
% <OR> manually copy in the resultant .bbl file
% set second argument of \begin to the number of references
% (used to reserve space for the reference number labels box)
\begin{thebibliography}{1}

\bibitem{IEEEhowto:kopka}
H.~Kopka and P.~W. Daly, \emph{A Guide to \LaTeX}, 3rd~ed.\hskip 1em plus
  0.5em minus 0.4em\relax Harlow, England: Addison-Wesley, 1999.

\end{thebibliography}

% biography section
% 
% If you have an EPS/PDF photo (graphicx package needed) extra braces are
% needed around the contents of the optional argument to biography to prevent
% the LaTeX parser from getting confused when it sees the complicated
% \includegraphics command within an optional argument. (You could create
% your own custom macro containing the \includegraphics command to make things
% simpler here.)
%\begin{biography}[{\includegraphics[width=1in,height=1.25in,clip,keepaspectratio]{mshell}}]{Michael Shell}
% or if you just want to reserve a space for a photo:

\begin{IEEEbiography}[{\includegraphics[width=1in,height=1.25in,clip,keepaspectratio]{picture}}]{John Doe}
\blindtext
\end{IEEEbiography}

% You can push biographies down or up by placing
% a \vfill before or after them. The appropriate
% use of \vfill depends on what kind of text is
% on the last page and whether or not the columns
% are being equalized.

%\vfill

% Can be used to pull up biographies so that the bottom of the last one
% is flush with the other column.
%\enlargethispage{-5in}




% that's all folks
\end{document}


